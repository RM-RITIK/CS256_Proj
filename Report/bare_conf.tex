%% Abstract and Introduction: Ikbal Singh
%% Background: Rashmi
%% Methodology: Ritik 
%% Results: Ritik 
%%Discussion: Ikbal Singh 
%%Conclusion: Rashmi 


%% bare_conf.tex
%% V1.4b
%% 2015/08/26
%% by Michael Shell
%% See:
%% http://www.michaelshell.org/
%% for current contact information.
%%
%% This is a skeleton file demonstrating the use of IEEEtran.cls
%% (requires IEEEtran.cls version 1.8b or later) with an IEEE
%% conference paper.
%%
%% Support sites:
%% http://www.michaelshell.org/tex/ieeetran/
%% http://www.ctan.org/pkg/ieeetran
%% and
%% http://www.ieee.org/

%%*************************************************************************
%% Legal Notice:
%% This code is offered as-is without any warranty either expressed or
%% implied; without even the implied warranty of MERCHANTABILITY or
%% FITNESS FOR A PARTICULAR PURPOSE! 
%% User assumes all risk.
%% In no event shall the IEEE or any contributor to this code be liable for
%% any damages or losses, including, but not limited to, incidental,
%% consequential, or any other damages, resulting from the use or misuse
%% of any information contained here.
%%
%% All comments are the opinions of their respective authors and are not
%% necessarily endorsed by the IEEE.
%%
%% This work is distributed under the LaTeX Project Public License (LPPL)
%% ( http://www.latex-project.org/ ) version 1.3, and may be freely used,
%% distributed and modified. A copy of the LPPL, version 1.3, is included
%% in the base LaTeX documentation of all distributions of LaTeX released
%% 2003/12/01 or later.
%% Retain all contribution notices and credits.
%% ** Modified files should be clearly indicated as such, including  **
%% ** renaming them and changing author support contact information. **
%%*************************************************************************


% *** Authors should verify (and, if needed, correct) their LaTeX system  ***
% *** with the testflow diagnostic prior to trusting their LaTeX platform ***
% *** with production work. The IEEE's font choices and paper sizes can   ***
% *** trigger bugs that do not appear when using other class files.       ***                          ***
% The testflow support page is at:
% http://www.michaelshell.org/tex/testflow/


\documentclass[conference]{IEEEtran}

\usepackage{graphicx}
\usepackage{amsmath,amsthm, amsfonts, amssymb, amsxtra, amsopn}
\usepackage{booktabs}
\newcommand{\RNum}[1]{\uppercase\expandafter{\romannumeral #1\relax}} 
% Some Computer Society conferences also require the compsoc mode option,
% but others use the standard conference format.
%
% If IEEEtran.cls has not been installed into the LaTeX system files,
% manually specify the path to it like:
% \documentclass[conference]{../sty/IEEEtran}





% Some very useful LaTeX packages include:
% (uncomment the ones you want to load)


% *** MISC UTILITY PACKAGES ***
%
%\usepackage{ifpdf}
% Heiko Oberdiek's ifpdf.sty is very useful if you need conditional
% compilation based on whether the output is pdf or dvi.
% usage:
% \ifpdf
%   % pdf code
% \else
%   % dvi code
% \fi
% The latest version of ifpdf.sty can be obtained from:
% http://www.ctan.org/pkg/ifpdf
% Also, note that IEEEtran.cls V1.7 and later provides a builtin
% \ifCLASSINFOpdf conditional that works the same way.
% When switching from latex to pdflatex and vice-versa, the compiler may
% have to be run twice to clear warning/error messages.






% *** CITATION PACKAGES ***
%
%\usepackage{cite}
% cite.sty was written by Donald Arseneau
% V1.6 and later of IEEEtran pre-defines the format of the cite.sty package
% \cite{} output to follow that of the IEEE. Loading the cite package will
% result in citation numbers being automatically sorted and properly
% "compressed/ranged". e.g., [1], [9], [2], [7], [5], [6] without using
% cite.sty will become [1], [2], [5]--[7], [9] using cite.sty. cite.sty's
% \cite will automatically add leading space, if needed. Use cite.sty's
% noadjust option (cite.sty V3.8 and later) if you want to turn this off
% such as if a citation ever needs to be enclosed in parenthesis.
% cite.sty is already installed on most LaTeX systems. Be sure and use
% version 5.0 (2009-03-20) and later if using hyperref.sty.
% The latest version can be obtained at:
% http://www.ctan.org/pkg/cite
% The documentation is contained in the cite.sty file itself.






% *** GRAPHICS RELATED PACKAGES ***
%
\ifCLASSINFOpdf
  % \usepackage[pdftex]{graphicx}
  % declare the path(s) where your graphic files are
  % \graphicspath{{../pdf/}{../jpeg/}}
  % and their extensions so you won't have to specify these with
  % every instance of \includegraphics
  % \DeclareGraphicsExtensions{.pdf,.jpeg,.png}
\else
  % or other class option (dvipsone, dvipdf, if not using dvips). graphicx
  % will default to the driver specified in the system graphics.cfg if no
  % driver is specified.
  % \usepackage[dvips]{graphicx}
  % declare the path(s) where your graphic files are
  % \graphicspath{{../eps/}}
  % and their extensions so you won't have to specify these with
  % every instance of \includegraphics
  % \DeclareGraphicsExtensions{.eps}
\fi
% graphicx was written by David Carlisle and Sebastian Rahtz. It is
% required if you want graphics, photos, etc. graphicx.sty is already
% installed on most LaTeX systems. The latest version and documentation
% can be obtained at: 
% http://www.ctan.org/pkg/graphicx
% Another good source of documentation is "Using Imported Graphics in
% LaTeX2e" by Keith Reckdahl which can be found at:
% http://www.ctan.org/pkg/epslatex
%
% latex, and pdflatex in dvi mode, support graphics in encapsulated
% postscript (.eps) format. pdflatex in pdf mode supports graphics
% in .pdf, .jpeg, .png and .mps (metapost) formats. Users should ensure
% that all non-photo figures use a vector format (.eps, .pdf, .mps) and
% not a bitmapped formats (.jpeg, .png). The IEEE frowns on bitmapped formats
% which can result in "jaggedy"/blurry rendering of lines and letters as
% well as large increases in file sizes.
%
% You can find documentation about the pdfTeX application at:
% http://www.tug.org/applications/pdftex





% *** MATH PACKAGES ***
%
%\usepackage{amsmath}
% A popular package from the American Mathematical Society that provides
% many useful and powerful commands for dealing with mathematics.
%
% Note that the amsmath package sets \interdisplaylinepenalty to 10000
% thus preventing page breaks from occurring within multiline equations. Use:
%\interdisplaylinepenalty=2500
% after loading amsmath to restore such page breaks as IEEEtran.cls normally
% does. amsmath.sty is already installed on most LaTeX systems. The latest
% version and documentation can be obtained at:
% http://www.ctan.org/pkg/amsmath





% *** SPECIALIZED LIST PACKAGES ***
%
%\usepackage{algorithmic}
% algorithmic.sty was written by Peter Williams and Rogerio Brito.
% This package provides an algorithmic environment fo describing algorithms.
% You can use the algorithmic environment in-text or within a figure
% environment to provide for a floating algorithm. Do NOT use the algorithm
% floating environment provided by algorithm.sty (by the same authors) or
% algorithm2e.sty (by Christophe Fiorio) as the IEEE does not use dedicated
% algorithm float types and packages that provide these will not provide
% correct IEEE style captions. The latest version and documentation of
% algorithmic.sty can be obtained at:
% http://www.ctan.org/pkg/algorithms
% Also of interest may be the (relatively newer and more customizable)
% algorithmicx.sty package by Szasz Janos:
% http://www.ctan.org/pkg/algorithmicx




% *** ALIGNMENT PACKAGES ***
%
%\usepackage{array}
% Frank Mittelbach's and David Carlisle's array.sty patches and improves
% the standard LaTeX2e array and tabular environments to provide better
% appearance and additional user controls. As the default LaTeX2e table
% generation code is lacking to the point of almost being broken with
% respect to the quality of the end results, all users are strongly
% advised to use an enhanced (at the very least that provided by array.sty)
% set of table tools. array.sty is already installed on most systems. The
% latest version and documentation can be obtained at:
% http://www.ctan.org/pkg/array


% IEEEtran contains the IEEEeqnarray family of commands that can be used to
% generate multiline equations as well as matrices, tables, etc., of high
% quality.




% *** SUBFIGURE PACKAGES ***
%\ifCLASSOPTIONcompsoc
%  \usepackage[caption=false,font=normalsize,labelfont=sf,textfont=sf]{subfig}
%\else
%  \usepackage[caption=false,font=footnotesize]{subfig}
%\fi
% subfig.sty, written by Steven Douglas Cochran, is the modern replacement
% for subfigure.sty, the latter of which is no longer maintained and is
% incompatible with some LaTeX packages including fixltx2e. However,
% subfig.sty requires and automatically loads Axel Sommerfeldt's caption.sty
% which will override IEEEtran.cls' handling of captions and this will result
% in non-IEEE style figure/table captions. To prevent this problem, be sure
% and invoke subfig.sty's "caption=false" package option (available since
% subfig.sty version 1.3, 2005/06/28) as this is will preserve IEEEtran.cls
% handling of captions.
% Note that the Computer Society format requires a larger sans serif font
% than the serif footnote size font used in traditional IEEE formatting
% and thus the need to invoke different subfig.sty package options depending
% on whether compsoc mode has been enabled.
%
% The latest version and documentation of subfig.sty can be obtained at:
% http://www.ctan.org/pkg/subfig




% *** FLOAT PACKAGES ***
%
%\usepackage{fixltx2e}
% fixltx2e, the successor to the earlier fix2col.sty, was written by
% Frank Mittelbach and David Carlisle. This package corrects a few problems
% in the LaTeX2e kernel, the most notable of which is that in current
% LaTeX2e releases, the ordering of single and double column floats is not
% guaranteed to be preserved. Thus, an unpatched LaTeX2e can allow a
% single column figure to be placed prior to an earlier double column
% figure.
% Be aware that LaTeX2e kernels dated 2015 and later have fixltx2e.sty's
% corrections already built into the system in which case a warning will
% be issued if an attempt is made to load fixltx2e.sty as it is no longer
% needed.
% The latest version and documentation can be found at:
% http://www.ctan.org/pkg/fixltx2e


%\usepackage{stfloats}
% stfloats.sty was written by Sigitas Tolusis. This package gives LaTeX2e
% the ability to do double column floats at the bottom of the page as well
% as the top. (e.g., "\begin{figure*}[!b]" is not normally possible in
% LaTeX2e). It also provides a command:
%\fnbelowfloat
% to enable the placement of footnotes below bottom floats (the standard
% LaTeX2e kernel puts them above bottom floats). This is an invasive package
% which rewrites many portions of the LaTeX2e float routines. It may not work
% with other packages that modify the LaTeX2e float routines. The latest
% version and documentation can be obtained at:
% http://www.ctan.org/pkg/stfloats
% Do not use the stfloats baselinefloat ability as the IEEE does not allow
% \baselineskip to stretch. Authors submitting work to the IEEE should note
% that the IEEE rarely uses double column equations and that authors should try
% to avoid such use. Do not be tempted to use the cuted.sty or midfloat.sty
% packages (also by Sigitas Tolusis) as the IEEE does not format its papers in
% such ways.
% Do not attempt to use stfloats with fixltx2e as they are incompatible.
% Instead, use Morten Hogholm'a dblfloatfix which combines the features
% of both fixltx2e and stfloats:
%
% \usepackage{dblfloatfix}
% The latest version can be found at:
% http://www.ctan.org/pkg/dblfloatfix




% *** PDF, URL AND HYPERLINK PACKAGES ***
%
%\usepackage{url}
% url.sty was written by Donald Arseneau. It provides better support for
% handling and breaking URLs. url.sty is already installed on most LaTeX
% systems. The latest version and documentation can be obtained at:
% http://www.ctan.org/pkg/url
% Basically, \url{my_url_here}.




% *** Do not adjust lengths that control margins, column widths, etc. ***
% *** Do not use packages that alter fonts (such as pslatex).         ***
% There should be no need to do such things with IEEEtran.cls V1.6 and later.
% (Unless specifically asked to do so by the journal or conference you plan
% to submit to, of course. )


% correct bad hyphenation here
\hyphenation{op-tical net-works semi-conduc-tor}


\begin{document}
%
% paper title
% Titles are generally capitalized except for words such as a, an, and, as,
% at, but, by, for, in, nor, of, on, or, the, to and up, which are usually
% not capitalized unless they are the first or last word of the title.
% Linebreaks \\ can be used within to get better formatting as desired.
% Do not put math or special symbols in the title.
\title{Reddit Self-Post Classification}


% author names and affiliations
% use a multiple column layout for up to three different
% affiliations
\author{\IEEEauthorblockN{Ritik Mehta}
\IEEEauthorblockA{Department of Computer Science \\
San Jose State Uninversity\\
ritik.mehta@sjsu.edu}
\and
\IEEEauthorblockN{Ikbal Singh Dhanjal}
\IEEEauthorblockA{Department of Computer Science\\
San Jose State Uninversity\\
ikbalsinghgurdevsingh.dhanjal@sjsu.edu}
\and
\IEEEauthorblockN{James Kirk\\ and Montgomery Scott}
\IEEEauthorblockA{Starfleet Academy\\
San Francisco, California 96678--2391\\
Telephone: (800) 555--1212\\
Fax: (888) 555--1212}}

% conference papers do not typically use \thanks and this command
% is locked out in conference mode. If really needed, such as for
% the acknowledgment of grants, issue a \IEEEoverridecommandlockouts
% after \documentclass

% for over three affiliations, or if they all won't fit within the width
% of the page, use this alternative format:
% 
%\author{\IEEEauthorblockN{Michael Shell\IEEEauthorrefmark{1},
%Homer Simpson\IEEEauthorrefmark{2},
%James Kirk\IEEEauthorrefmark{3}, 
%Montgomery Scott\IEEEauthorrefmark{3} and
%Eldon Tyrell\IEEEauthorrefmark{4}}
%\IEEEauthorblockA{\IEEEauthorrefmark{1}School of Electrical and Computer Engineering\\
%Georgia Institute of Technology,
%Atlanta, Georgia 30332--0250\\ Email: see http://www.michaelshell.org/contact.html}
%\IEEEauthorblockA{\IEEEauthorrefmark{2}Twentieth Century Fox, Springfield, USA\\
%Email: homer@thesimpsons.com}
%\IEEEauthorblockA{\IEEEauthorrefmark{3}Starfleet Academy, San Francisco, California 96678-2391\\
%Telephone: (800) 555--1212, Fax: (888) 555--1212}
%\IEEEauthorblockA{\IEEEauthorrefmark{4}Tyrell Inc., 123 Replicant Street, Los Angeles, California 90210--4321}}




% use for special paper notices
%\IEEEspecialpapernotice{(Invited Paper)}




% make the title area
\maketitle

% As a general rule, do not put math, special symbols or citations
% in the abstract
\begin{abstract}
  Reddit, a prominent social media platform hosts a diverse array of discussions across numerous subreddits. The users are required to add a flair or genre when creating a post. This paper performs classification of self-posts into different flairs which can be used for automatically flairing posts. It reduces the problem into a text classification problem and explores the effectiveness of various models including Logistic Regression, Random Forest, Long Short-Term Memory (LSTM) networkss, and LSTM-Convolutional Neural Network (CNN-LSTM) hybrids. Our experiments involve utilizing different word embeddings such as TF-IDF, Word2Vec, and BERT, to capture semantic features for classification. We evaluate the performance of these models on Reddit dataset using accuracy. 
\end{abstract}

% no keywords




% For peer review papers, you can put extra information on the cover
% page as needed:
% \ifCLASSOPTIONpeerreview
% \begin{center} \bfseries EDICS Category: 3-BBND \end{center}
% \fi
%
% For peerreview papers, this IEEEtran command inserts a page break and
% creates the second title. It will be ignored for other modes.
\IEEEpeerreviewmaketitle



\section{Introduction}
% no \IEEEPARstart
Reddit is a social media website which combines anonymity with the popularity of the early internet forums. It allows users to create their own subreddits based on a topic and then users can join those subreddits and make posts in the subreddits. In 2019, Reddit beat Twitter in terms of Monthly Active Userbase (MAU) and in 2024, 1 in every 7 humans visit Reddit at-least once a month. These subreddits are moderated by users who are passionate to keep the community going without any financial inventive. One of the roles of the moderators is to effectively manage and curate the posts made by the users in the subreddit. Often times this is achieved by something like "flairs". Flairs are a categorization of posts by a genre. For example, in the subreddit r/help, there are multiple flairs like "Access", "Profile", "Reddit Premium", "Mobile/App", etc. It's expected, while not necessarily mandatory, from the users to add the flair while creating a post. However, many posts are not created with a flair. So the moderators have to either add the flairs themselves or they need to auto-remove the post by using a bot. This causes extra work on moderators and may sometimes impact the productive discussion in the subreddit. 

This paper aims to solve the problem of Reddit Self-Post Classification into different genres or flairs. This problem falls under Text Classification in Natural Language Processing. We solve it using Supervised Learning methods. Section \RNum{2} explains the background and a brief literature survey of the state-of-the-art Text Classification models used. Section \RNum{3} proposes the methodology used, including the embeddings and the models. Section \RNum{4} presents the results obtained from training and \RNum{5} outlines the interpretation of the results and challenges. Finally \RNum{6} concludes with the future scope and possible improvements which can be made to the work. 

% % You must have at least 2 lines in the paragraph with the drop letter
% % (should never be an issue)
% I wish you the best of success.

% \hfill mds
 
% \hfill August 26, 2015

% \subsection{Subsection Heading Here}
% Subsection text here.


% \subsubsection{Subsubsection Heading Here}
% Subsubsection text here.


% An example of a floating figure using the graphicx package.
% Note that \label must occur AFTER (or within) \caption.
% For figures, \caption should occur after the \includegraphics.
% Note that IEEEtran v1.7 and later has special internal code that
% is designed to preserve the operation of \label within \caption
% even when the captionsoff option is in effect. However, because
% of issues like this, it may be the safest practice to put all your
% \label just after \caption rather than within \caption{}.
%
% Reminder: the "draftcls" or "draftclsnofoot", not "draft", class
% option should be used if it is desired that the figures are to be
% displayed while in draft mode.
%
%\begin{figure}[!t]
%\centering
%\includegraphics[width=2.5in]{myfigure}
% where an .eps filename suffix will be assumed under latex, 
% and a .pdf suffix will be assumed for pdflatex; or what has been declared
% via \DeclareGraphicsExtensions.
%\caption{Simulation results for the network.}
%\label{fig_sim}
%\end{figure}

% Note that the IEEE typically puts floats only at the top, even when this
% results in a large percentage of a column being occupied by floats.


% An example of a double column floating figure using two subfigures.
% (The subfig.sty package must be loaded for this to work.)
% The subfigure \label commands are set within each subfloat command,
% and the \label for the overall figure must come after \caption.
% \hfil is used as a separator to get equal spacing.
% Watch out that the combined width of all the subfigures on a 
% line do not exceed the text width or a line break will occur.
%
%\begin{figure*}[!t]
%\centering
%\subfloat[Case I]{\includegraphics[width=2.5in]{box}%
%\label{fig_first_case}}
%\hfil
%\subfloat[Case II]{\includegraphics[width=2.5in]{box}%
%\label{fig_second_case}}
%\caption{Simulation results for the network.}
%\label{fig_sim}
%\end{figure*}
%
% Note that often IEEE papers with subfigures do not employ subfigure
% captions (using the optional argument to \subfloat[]), but instead will
% reference/describe all of them (a), (b), etc., within the main caption.
% Be aware that for subfig.sty to generate the (a), (b), etc., subfigure
% labels, the optional argument to \subfloat must be present. If a
% subcaption is not desired, just leave its contents blank,
% e.g., \subfloat[].


% An example of a floating table. Note that, for IEEE style tables, the
% \caption command should come BEFORE the table and, given that table
% captions serve much like titles, are usually capitalized except for words
% such as a, an, and, as, at, but, by, for, in, nor, of, on, or, the, to
% and up, which are usually not capitalized unless they are the first or
% last word of the caption. Table text will default to \footnotesize as
% the IEEE normally uses this smaller font for tables.
% The \label must come after \caption as always.
%
%\begin{table}[!t]
%% increase table row spacing, adjust to taste
%\renewcommand{\arraystretch}{1.3}
% if using array.sty, it might be a good idea to tweak the value of
% \extrarowheight as needed to properly center the text within the cells
%\caption{An Example of a Table}
%\label{table_example}
%\centering
%% Some packages, such as MDW tools, offer better commands for making tables
%% than the plain LaTeX2e tabular which is used here.
%\begin{tabular}{|c||c|}
%\hline
%One & Two\\
%\hline
%Three & Four\\
%\hline
%\end{tabular}
%\end{table}


% Note that the IEEE does not put floats in the very first column
% - or typically anywhere on the first page for that matter. Also,
% in-text middle ("here") positioning is typically not used, but it
% is allowed and encouraged for Computer Society conferences (but
% not Computer Society journals). Most IEEE journals/conferences use
% top floats exclusively. 
% Note that, LaTeX2e, unlike IEEE journals/conferences, places
% footnotes above bottom floats. This can be corrected via the
% \fnbelowfloat command of the stfloats package.

\section{Background}

Text classification has become a fundamental task in natural language processing (NLP), driven by the exponential growth of textual data on the internet. Various machine learning and deep learning techniques have been developed to classify texts into predefined categories efficiently and accurately. This survey explores the evolution of text classification methodologies, comparing traditional approaches with modern deep learning models and evaluating their performance across different datasets and applications.

Early text classification models relied on statistical and machine learning techniques, such as Naive Bayes (NB), k-Nearest Neighbors (k-NN), and Support Vector Machines (SVM). These models typically employed feature extraction methods like Term Frequency-Inverse Document Frequency (TF-IDF) to convert text into numerical representations suitable for classification.

Naive Bayes classifiers, based on Bayes' theorem, are widely used for their simplicity and effectiveness in text classification tasks, including spam detection and sentiment analysis [1]. The k-NN algorithm, though intuitive, is computationally intensive and often impractical for large datasets [2]. SVM, on the other hand, constructs hyperplanes in a multidimensional space to separate different classes and has shown robust performance in many text classification tasks [3].

The advent of deep learning has revolutionized text classification, enabling models to automatically learn feature representations from raw text data. Recurrent Neural Networks (RNNs) and their variants, such as Long Short-Term Memory (LSTM) networks, have been particularly effective for sequential data processing.

LSTM networks address the limitations of traditional RNNs, such as the vanishing gradient problem, by incorporating memory cells that maintain long-term dependencies [4]. These models have achieved state-of-the-art results in various NLP tasks, including sentiment analysis, language modeling, and sequence-to-sequence predictions [5].

Transformers, particularly the Bidirectional Encoder Representations from Transformers (BERT), have set new benchmarks in text classification. BERT leverages a bidirectional training approach and self-attention mechanisms to capture contextual information from both left and right contexts in a sentence, significantly improving the performance of text classification models on numerous datasets [6].

BERT-based models, such as BERT4TC, have been further refined to enhance text classification, often incorporating domain-specific knowledge and auxiliary sentences to improve their effectiveness in specific tasks [7]. These advancements have led to significant improvements in various NLP applications, including aspect-based sentiment analysis and question answering [8].

Several studies have compared the performance of traditional and deep learning-based text classification methods. A comprehensive review of over 150 deep learning models highlighted their superior performance over classical approaches in tasks such as sentiment analysis, news categorization, and question answering [9]. Another study specifically compared the TF-IDF and Word2Vec models, concluding that TF-IDF combined with SVM achieved higher accuracy in emotion text classification [2].

\section{Methodology}
In this section, we begin by presenting the dataset employed in our experiments, followed by an overview of our project's experimental design.

\subsection{Dataset and Preprocessing}
Our dataset is publicly available at~\cite{dataset}. It consists of over one million Reddit self-posts among 39 genres. Due to constraints on the computation power, we took 3000 samples from each genre to reduce the training time, and employed the following preprocessing techniques:
\begin{itemize}
    \item Text in the title, selftext, and genre was converted to lowercase.
    \item Commas, periods, and text within and including angle brackets were removed from the title and selftext.
    \item Question marks and exclamation points were wrapped in white space.
    \item The title and selftext were concatenated to obtain the feature vectors and the corresponding genre was regarded as the target label.
\end{itemize}

\begin{figure}[!htb]
\centering
\includegraphics[width=90mm, height = 30mm]{Images/DataPreprocessing.png}
\caption{Flow of Data}\label{fig:dataPreprocessing}
\end{figure}

\subsection{Word Embeddings}
We used three popular techniques to generate word embeddings of concatenated strings obtained in the previous step:

\begin{itemize}
    \item \textbf{TF-IDF Vectorizer}: TF-IDF Vectorizer~\cite{tfidf} assigns each word in the document a weight based on Term Frequency (TF) and Inverse Document Frequency (IDF). Term Frequency measures how often a term appears in a document. The more often a term appears in a document, the higher its weight. Inverse Document Frequency (IDF) measures how important a term is within the entire corpus. Terms that appear frequently across many documents are given lower weights.
    \item \textbf{Word2Vec}: Word2Vec~\cite{word2vec} transforms words into dense vectors within a high-dimensional space. These vectors encode the semantic meaning of words, enabling machines to comprehend relationships between them. When training Word2Vec on our own corpus, we set the \textit{vector\_size} to 300 and \textit{window} to 5. The first parameter specifies the dimensionality of the word vectors and the second parameter determines the context window size. 
    \item \textbf{BERT}: BERT (Bidirectional Encoder Representations from Transformers)~\cite{devlin2019bert} is a natural language processing model developed by Google. It uses self-attention mechanisms to capture relationships between words in a sentence. Unlike traditional word embedding techniques like Word2Vec, BERT considers the context of each word in a sentence bidirectionally.
\end{itemize}

\subsection{Logistic Regression}
We trained a logistic regression model~\cite{Komarek-2004-8925} with all three embeddings generated in the previous step. Since, Word2Vec and BERT embeddings are generated for a word, we took an average of all the word embeddings for each sample to get a single vector representation. As a result, we obtained a feature vector of length 300 corresponding to each data sample. For hyper-parameter tuning, we considered the following hyper-parameters:

\begin{itemize}
    \item \textbf{\textit{C}}: It is the inverse of regularization strength. Smaller values specify stronger regularization. Regularization helps prevent overfitting by adding a penalty term to the loss function. We test our logistic regression model with~$C: {0.01, 0.1, 1}$.
    \item \textbf{\textit{solver}}: It is the algorithm to use in the optimization problem. Without going into much detail, we tested our logistic regression model with the following values of $solver: {'lbfgs', 'sag', 'saga'}$
\end{itemize}

\subsection{Random Forest}
Random Forest was originally proposed in~\cite{randomforestorig}. 
It comprises collections of decision trees. A decision tree is a basic supervised learning technique used for classification or prediction based on a tree-like structure. It consists of a root node, branches, internal nodes, and leaf nodes, and can be seen as a series of \texttt{if}-\texttt{else} statements. We trained a random forest model with all three embeddings, using the same feature vectors used for the logistic regression. The hyperparameter considered for our random forest model is as follows:
\begin{itemize}
    \item \textbf{\textit{max\_depth}}: It represents the maximum depth of a decision tree. We tested this hyperparameter with the following values: $max\_depth: {5, 7, 10}$
    \item \textbf{\textit{n\_estimators}}: It represents the number of decision trees in the random forest. We tested this hyperparameter with the following values: $n\_estimator: {150, 200, 400}$
\end{itemize}

\subsection{Long Short-Term Memory (LSTM)}
LSTM~\cite{lstm}, a recurrent neural network (RNN) architecture, addresses the vanishing gradient problem that impedes traditional RNNs from effectively learning long sequences. LSTMs are specifically designed to capture long-term dependencies in sequential data by maintaining a memory state over time.

An LSTM comprises three gates to regulate the flow of information: the input gate, the forget gate, and the output gate. The input gate regulates the flow of new information into the memory cell, the forget gate determines which information should be discarded from the cell, and the output gate controle the exposure of the internal state. 

Each of our LSTM model comprises of 64 memory units, and also consists of a Dropout Layer and a Dense Layer with softmax activation. In the Dropout Layer, 10\% of the input units will be randomly set to 0 during each training epoch.

\subsection{LSTM-CNN}
The LSTM-CNN model is a combination of LSTM and Convolutional Neural Network~\cite{cnn}. The training procedure of our LSTM model can be described as:
\begin{itemize}
    \item \textbf{Generating Embeddings (or Embeddings Layer)}: Unlike the three baseline models that we discussed above, we don't take an average of all the word embeddings in a data sample. We first calculate the median number of words present in each sample, which comes out to be~$69$. If the number of words in a data sample is less than the median length, we pad the data sample, and generate word embeddings corresponding to each word in a data sample. For example, in case of BERT, corresponding to each data sample, we get a feature vector of dimensions~$(69, 768)$.
    \item \textbf{Convolutional Layer}: The feature vectors generated in the previous step are passed to the convolutional layer with ReLU activation function and 128 filters.
    \item \textbf{\textit{Max pooling Layer}}: The output from the convolutional layer is passed to the max pooling layer with the pool size of 2.
    \item \textbf{Spatial Dropout}: The next layer in our proposed model is a spatial dropout layer. Spatial dropout is a regularization technique employed to aid our model in generalization and to prevent overfitting. We apply a dropout rate of 0.2.
    \item \textbf{LSTM}: The number of LSTM units are set to the embedding dimension. For example, the number of LSTM units in the LSTM layer of BERT-LSTM-CNN is 768. Both dropout and recurrent dropout rates are set to 0.2
    \item \textbf{Dense}: This layer is fully connected, and it utilizes a softmax activation function to output the probability distribution across all classes.
\end{itemize}
\section{Results}
In this section, we present the results obtained from the techniques discussed in the previous section.

\subsection{Logistic Regression}
Table~\ref{tab: lr_models} shows the accuracy of our Logistic Regression models. We observed that the Logistic Regression model provides best accuracy of 68.79\% using TF-IDF Vectors. The confusion matrix for same is shown in Figure~\ref{fig:lr}.

\begin{table}[!htb]
\centering
\caption{Accuracy of Logistic Regression Models}
\label{tab: lr_models}
\begin{tabular}{c|c}\midrule\midrule
Model & Accuracy\\ \midrule
\textbf{TF-IDF + Logistic Regression} & \textbf{68.79\%} \\
Word2Vec + Logistic Regression & 40.18\% \\
BERT + Logistic Regression & 56.83\% \\ \midrule\midrule
\end{tabular}
\end{table}

\begin{figure}[!htb]
\centering
\includegraphics[width=70mm, height = 60mm]{Images/Logistic Regression.png}
\caption{Confusion Matrix for TF-IDF + Logistic Regression}\label{fig:lr}
\end{figure}

\subsection{Random Forest}

Table~\ref{tab: rf_models} shows the accuracy of our Random Forest models. We observed that the Random Forest model provides best accuracy of 45.58\% using TF-IDF Vectors. The confusion matrix for same is shown in Figure~\ref{fig:rf}.

\begin{table}[!htb]
\centering
\caption{Accuracy of Random Forest Models}
\label{tab: rf_models}
\begin{tabular}{c|c}\midrule\midrule
Model & Accuracy\\ \midrule
\textbf{TF-IDF + Random Forest} & \textbf{45.58\%} \\
Word2Vec + Random Forest & 27.14\% \\
BERT + Random Forest & 42.36\% \\ \midrule\midrule
\end{tabular}
\end{table}

\begin{figure}[!htb]
\centering
\includegraphics[width=70mm, height = 60mm]{Images/Random Forest.png}
\caption{Confusion Matrix for TF-IDF + Random Forest}\label{fig:rf}
\end{figure}

\subsection{Long Short-Term Memory (LSTM)}

Table~\ref{tab: lstm_models} shows the accuracy of our LSTM models. We observed that the LSTM model provides best accuracy of 59.09\% using TF-IDF Vectors. The confusion matrix for same is shown in Figure~\ref{fig:lstm}.

\begin{table}[!htb]
\centering
\caption{Accuracy of LSTM Models}
\label{tab: lstm_models}
\begin{tabular}{c|c}\midrule\midrule
Model & Accuracy\\ \midrule
\textbf{TF-IDF + LSTM} & \textbf{59.09\%} \\
Word2Vec + LSTM & 37.02\% \\
BERT + LSTM & 54.10\% \\ \midrule\midrule
\end{tabular}
\end{table}

\begin{figure}[!htb]
\centering
\includegraphics[width=70mm, height = 60mm]{Images/LSTM.png}
\caption{Confusion Matrix for TF-IDF + LSTM}\label{fig:lstm}
\end{figure}

\subsection{LSTM-CNN}

Table~\ref{tab: lstm_cnn_models} shows the accuracy of our LSTM-CNN models. We observed that the LSTM-CNN model provides best accuracy of 61.53\% using BERT embeddings. The confusion matrix for same is shown in Figure~\ref{fig:lstm_cnn}.

\begin{table}[!htb]
\centering
\caption{Accuracy of LSTM-CNN Models}
\label{tab: lstm_cnn_models}
\begin{tabular}{c|c}\midrule\midrule
Model & Accuracy\\ \midrule
TF-IDF + LSTM-CNN & 59.51\% \\
Word2Vec + LSTM-CNN & 47.21\% \\
\textbf{BERT + LSTM-CNN} & \textbf{61.53}\% \\ \midrule\midrule
\end{tabular}
\end{table}

\begin{figure}[!htb]
\centering
\includegraphics[width=70mm, height = 60mm]{Images/LSTM-CNN.png}
\caption{Confusion Matrix for TF-IDF + LSTM-CNN}\label{fig:lstm_cnn}
\end{figure}

\section{Discussion}

The results section presents a comprehensive analysis of various machine learning models applied to the dataset. In this section, we will focus on interpreting these results, comparing different models, addressing limitations, and challenges observed during the training.

Starting with Logistic Regression, the TF-IDF approach yielded the highest accuracy among the models discussed, reaching 68.79\%. This result is notable and suggests that the TF-IDF technique effectively captured the features necessary for classification in this context. However, it's crucial to delve deeper into why Word2Vec and BERT did not perform as well in this scenario, as understanding these nuances can inform future model selection. One of the possible reasons for Word2Vec is that we did not fine-tune Word2Vec along with pre-trained weights. The paper uses the Word2Vec architecture and retrains it on the dataset corpus. 

Moving on to Random Forest, TF-IDF again showed the highest accuracy, although notably lower than Logistic Regression at 45.58\%. This discrepancy between the two algorithms raises questions about the dataset's characteristics and how different models handle its complexities.

The LSTM model achieved a respectable accuracy of 59.09\% with TF-IDF vectors, showcasing the strength of sequence modeling in this task. BERT, despite its pre-training on large corpora, fell slightly short of TF-IDF in the LSTM context. This is likely because when computing the sentence embeddings, we average the embeddings for each word to compute the sentence embedding. This may cause to lose the internal context between words in the sentence.

The LSTM-CNN hybrid model, particularly with BERT embeddings, demonstrated the highest accuracy at 61.53\%. It's interesting to see that a powerful architecture LSTM-CNN and a powerful embedding like BERT produces lower accuracy than a simple Logistic Regression with TF-IDF. We believe that it is because of the smaller corpus in the sampled dataset. These models should be able to exploit their explanatory power as we increase the dataset size and fine-tune using pre-trained weights for both Word2Vec and BERT.

\section{Conclusion}
The conclusion goes here.




% conference papers do not normally have an appendix


% use section* for acknowledgment
\section*{Acknowledgment}


The authors would like to thank...

\begin{thebibliography}{1}

\bibitem{IEEEhowto:naik}
P. Naik, S. A. S., and V. Kuri, \emph{"Classification of posts on Reddit,"} Graduate Student, CSE Dept, UCSD, CA, USA, 2021
\bibitem{IEEEhowto:cahyani}
D. E. Cahyani and I. Patasik, \emph{"Performance comparison of TF-IDF and Word2Vec models for emotion text classification,"}, Bulletin of Electrical Engineering and Informatics, vol. 10, no. 5, pp. 2780-2788, 2021.
\bibitem{IEEEhowto:luan}
Y. Luan and S. Lin, \emph{"Research on Text Classification Based on CNN and LSTM,"} IEEE International Conference on Artificial Intelligence and Computer Applications, 2019.
\bibitem{IEEEhowto:sun}
Y. Sun, L. Wang, Y. Li, S. Feng, H. Tian, and J. Wu, \emph{"Improving BERT-Based Text Classification with Auxiliary Sentences,"} IEEE Access, vol. 8, pp. 191563-191574, 2020.
\bibitem{IEEEhowto:kingma}
D. P. Kingma and J. Ba, \emph{"Text Classification Using Long Short-Term Memory (LSTM),"} ArXiv arXiv:1406.2661, 2014.
\bibitem{IEEEhowto:joulin}
A. Joulin, E. Grave, P. Bojanowski, and T. Mikolov, \emph{"Bag of Tricks for Efficient Text Classification,"} in Proceedings of the 15th Conference of the European Chapter of the Association for Computational Linguistics: Volume 2, Short Papers, pp. 427-431, 2017.
\bibitem{IEEEhowto:milad}
Milad Moradi, Georg Dorffner, Matthias Samwald,
\emph{"Deep contextualized embeddings for quantifying the informative content in biomedical text summarization"},Computer Methods and Programs in Biomedicine, Volume 184,2020,105117,ISSN 0169-2607,
\bibitem{IEEEhowto:zhang}
X. Zhang, J. Zhao, and Y. LeCun, \emph{"Character-level Convolutional Networks for Text Classification,"} in Advances in Neural Information Processing Systems, pp. 649-657, 2015
\bibitem{IEEEhowto:minaee}
Minaee, S., Kalchbrenner, N., Cambria, E., Nikzad, N., Chenaghlu, M., and  Gao, J. (2021). \emph{"Deep learning--based text classification: a comprehensive review."}, ACM computing surveys (CSUR), 54(3), 1-40.


\end{thebibliography}





% trigger a \newpage just before the given reference
% number - used to balance the columns on the last page
% adjust value as needed - may need to be readjusted if
% the document is modified later
%\IEEEtriggeratref{8}
% The "triggered" command can be changed if desired:
%\IEEEtriggercmd{\enlargethispage{-5in}}

% references section

% can use a bibliography generated by BibTeX as a .bbl file
% BibTeX documentation can be easily obtained at:
% http://mirror.ctan.org/biblio/bibtex/contrib/doc/
% The IEEEtran BibTeX style support page is at:
% http://www.michaelshell.org/tex/ieeetran/bibtex/
%\bibliographystyle{IEEEtran}
% argument is your BibTeX string definitions and bibliography database(s)
%\bibliography{IEEEabrv,../bib/paper}
%
% <OR> manually copy in the resultant .bbl file
% set second argument of \begin to the number of references
% (used to reserve space for the reference number labels box)




% that's all folks
\end{document}


